\chapter{Definiciones de las características a medir para caracterizar el movimiento}
\label{chap:definiciones}

Para poder clasificar las trayectorias de los espermatozoides, es necesario medir características que se definen a continuación.

\begin{itemize}
    \item \textbf{VCL} - "Velocidad curvilínea" ($\mu$m/seg): Es la velocidad promedio del espermatozoide sobre todo el camino verdadero. 
    \item \textbf{VSL} - "Velocidad linieal" ($\mu$m/seg): Es la velocidad promedio del espermatozoide sobre el camino en linea recta que une la primera y la ultima posición detectada.
    \item \textbf{VAP} - "Velocidad del camino promedio" ($\mu$m/seg): Se calcula como la velocidad promedio del espermatozoide sobre el camino promedio. Dado que el calculo del camino promedio puede variar en las distintas implementaciones dicho valor no es comparable.
    \item \textbf{ALH} - ''Amplitud del desplazamiento lateral de la cabeza" ($\mu$m): Mide la distancia entre el espermatozoide y el camino promedio. Puede ser expresada como el máximo o el promedio en la trayectoria. También puede variar dependiendo de la implementación, por lo tanto tampoco es comparable.
    \item \textbf{LIN} - "Linealidad" (sin dimensión): Se calcula como VSL/VCL.
    \item \textbf{STR} - "Rectitud" (sin dimensión): Se calcula como VAP/VCL.
    \item \textbf{MAD} - "Promedio de desplazamiento angular" (grados): El promedio de los valores absolutos del ángulo instantáneo de rotación del espermatozoide sobre su trayectoria curvilínea.
    \item \textbf{WOB} - ''Oscilación de la trayectoria" (sin dimensión): Es una medida de la oscilación del camino verdadero en relación con el camino promedio. Se calcula como VAP/VCL. 
    \item \textbf{BCF} - "Beat-cross frequency" (HZ): La frecuencia promedio en la cual el camino verdadero cruza el camino promedio.
\end{itemize}{}
