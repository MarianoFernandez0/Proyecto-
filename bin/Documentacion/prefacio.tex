\chapter{Prefacio}

El proyecto se realiza en el contexto del espacio interdisciplinario IMAGINA. Surge a raíz de la necesidad, por parte de la Dra. Rossana Sapiro del \textit{ Departamento de Histología y Embriología de la Facultad de Medicina}, de desarrollar una herramienta para procesamiento de imágenes con el fin de investigar una hipótesis propia. Esta hipótesis se basa en la posible existencia de una alta correlación entre el movimiento de los espermatozoides y la actividad mitocondrial indicada por el nivel de fluorescencia. De este modo, se busca desarrollar una herramienta capaz de realizar el \textit{tracking} y medir fluorescencia de espermatozoides, para que a partir de cada espermatozoide y su recorrido, se puedan obtener diversas medidas biológicas para estudiar la correlación con el nivel de fluorescencia.
Si bien en el \textit{ Departamento de Histología y Embriología de la Facultad de Medicina} ya cuenta con un software capaz de realizar \textit{tracking} de espermatozoides y clasificar los tipos de movimientos en cinco categorías, el mismo lo hace para imágenes de campo claro. Para la investigación este programa no puede ser utilizado dado que las imágenes que se utilizaran son imágenes de fluorescencia. De aquí surge la necesidad de construir una herramienta que  sea capaz de realizar el \textit{tracking} en imágenes de fluorescencia y que a su vez tenga el agregado de poder medir el nivel de fluorescencia de los espermatozoides en las mismas. De esta manera, se podría dar información del tipo de movimiento conjuntamente con la fluorescencia registrada en las imágenes, la cual será utilizada para que los expertos saquen conclusiones del punto de vista biológico.

\thispagestyle{plain}

\begin{flushright}
Lucía, Leonardo, Mariano
\end{flushright}
